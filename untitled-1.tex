
\documentclass[12pt]{article}
\usepackage{authblk}
\usepackage{authblk}
\usepackage{graphicx}
\graphicspath{ {} }


\begin{document}
\title{\textbf{Raport Final\\ Proiect Structuri de date}}
\date{}
\author{
  Bejan-Topse Cosmin\
  \texttt{}
  \and
  Drobnitchi Daniel\
  \texttt{}
	\and
	Suto Robert
}
\affil{Facultatea De Matematica si Informatica, Universitatea Bucuresti}
\maketitle
\newpage
  
\tableofcontents
\newpage
\section{Introducere}
Echipa noastra a ales sa implementeze un arbore AVL. \\ Arborii AVL sunt un model de arbori binari de cautare echilibrat dupa inaltime, care se reechilibreaza dupa fiecare inserare sau stergere.\\
La inserare se adauga nodul exact ca intr-un arbore binar de cautare, iar dupa se verifica factorul de balansare si se incepe sau nu balansarea lui (aceasta balansare se face cu rotatii duble).\\
Proprietatea de baza a unui arbore AVL este aceea ca, diferenta in modul dintre inaltimea nodului stang si cea a nodului drept este mai mica sau egala cu 1.\\ \\
Structura elementelor unui arbore echilibrat poate fi reprezentata astfel:

\begin{verbatim}
class AVL
{
public:
    int key;
    AVL *left;
    AVL *right;
    int height;
}

\end{verbatim}




\begin{itemize}
  \item key = valorea nodului
  \item left/right = pointeri catre copii (stanga, respectiv dreapta)
  \item height = inaltimea
\end{itemize}

\section{Analiza Timp a programului}
\subsection{Inaltime}

Putem arata ca un arbore AVL cu n noduri are inaltimea O(log n):\\
Fie N\textsubscript{h} minimul de noduri care pot forma un arbore AVL de inaltime h.\\
Daca stim N\textsubscript{h-1} si  N\textsubscript{h-2}, putem determina  N\textsubscript{h}. Stiind ca acest arbore cu N\textsubscript{h} noduri are inaltimea h, radacina are un fiu cu inaltimea h-1, adica
un sub-arbore cu N\textsubscript{h-1} noduri. Cunoscand proprietatile unui arbore AVL, daca un fiu are
inaltimea h-1, inaltimea minima a celuilalt fiu este h-2. Avand un sub-arbore drept cu N\textsubscript{h-1}
noduri si un sub-arbore stang cu N\textsubscript{h-2} noduri, am construit un arbore AVL cu inaltimea h
si cu numarul minim de noduri necesare pentru a alcatui acest arbore.
\\Acest arbore are in total N\textsubscript{h-1} + N\textsubscript{h-2} + 1 (radacina) noduri. Folosind aceasta formula
putem deduce:


\[ N_h = N\textsubscript{h-1} + N\textsubscript{h-2} + 1 \]
\[ N\textsubscript{h-1} = N\textsubscript{h-2} + N\textsubscript{h-3} + 1 \]
\[ N_h = ( N\textsubscript{h-2} + N\textsubscript{h-3} + 1 ) + N\textsubscript{h-2} + 1  \]
\[ N_h > 2N\textsubscript{h-2}\]
\[ N_h > 2\textsuperscript{h/2}\]
\[ log N_h > log 2\textsuperscript{h/2}\]
\[ 2log N_h > h\]
\[h = O(log N_h)\]

\subsection{Rotatii}

Pentru a mentine arborele balansat, avem nevoie de doua operatii: o rotatie la stanga si una la dreapta.
Rotatiile sunt simple rearanjari a nodurilor pentru a interschimba inaltimile in timp ce pastram
ordinea elementelor sale. \\O rotatie necesita o reatribuire a fiului drept, stang si al parintelui la cateva 
noduri, dar nimic mai mult de atat. Rotatiile sunt operatii de timp O(1).
Avem 4 tipuri de rotatii:


\begin{itemize}
  \item Dreapta - Dreapta (daca factorul de balans este negativ si factorul de balans al nodului drept este tot negativ)
\end{itemize}
\noindent\makebox[\textwidth]{\includegraphics[width=\textwidth]{rr_rotation}}
\newpage
\begin{itemize}

  \item Stanga - Stanga (daca factorul de balans este pozitiv si factorul de balans al nodului stang este tot pozitiv
\end{itemize}
\noindent\makebox[\textwidth]{\includegraphics[width=\textwidth]{ll_rotation}}
\begin{itemize}
  \item Dreapta - Stanga (daca factorul de balans este negativ si factorul de balans al nodului drept este pozitiv)
\end{itemize}
\noindent\makebox[\textwidth]{\includegraphics[width=\textwidth]{rl_rotation}}
\begin{itemize}
  \item Stanga - Dreapta (daca factorul de balans este pozitiv si factorul de balans al nodului stang este negativ)
\end{itemize}
\noindent\makebox[\textwidth]{\includegraphics[width=\textwidth]{lr_rotation}}


\newpage
\subsection{Inserare}
Inserarea unui nod intr-un AVL se realizeaza in doua etape:
\begin{enumerate}
\item Se adauga nodul intr-un arbore binar de cautare (se cauta pozitia de inserare pe baza comparatiei cheii nodului si cheia nodului stang si drept).
\item Se va actualiza inaltimea fiecarui nod de la pozitia inserarii pana la radacina, iar daca este necesar se aplica operatii de rotire (pentru a mentine proprietatea AVL - ului)
\end{enumerate}
Inserarile intr-un arbore AVL sunt inserarile din arborele binar de cautare, la care se adauga cel mult doua rotatii.
Stiind ca inserarile intr-un arbore binar de cautare au complexitatea O(h), rotatiile au O(1) si ca inaltimea arborelui AVL este h=O(log n), concluzionam ca inserarile in AVL necesita timpul O(log n).

\subsection{Stergere}
Operatia de stergere pentru o cheie dintr-un arbore AVL se aseamana cu stergerea unei chei dintr-un arbore binar de cautare. Aceasta stergere se poate rezuma in 3 etape
\begin{enumerate}

\item Cautarea cheii pe care dorim sa o stergem.
\item Stergerea nodului cu cheia respectiva. Daca nodul are 2 succesori, il vom inlocui cu cel mai mare nod din subarborele stang(sau cel mai mic nod din subarborele drept). Altfel il vom inlocui cu unul dintre succesorii sai nenuli, iar in cazul in care acesta este frunza, se va inlocui cu NULL.
\item Reactualizarea inaltimilor si realizarea rotatiilor necesare.

\end{enumerate}


Stiind ca stergerile intr-un arbore binar de cautare au complexitatea O(h), rotatiile au O(1) si ca inaltimea arborelui AVL este h=O(log n), ajungem la concluzia ca stergerile in AVL necesita timpul O(log n)

\newpage
\subsection{Valoarea minima\slash maxima }

Pentru aceste operatii, algoritmul se aseamana.\\
Astfel, pentru cea mai mica valoare, parcurgem recursiv de la radacina la stanga, pana cand valoarea left este NULL, iar pentru cea mai mare parcurgem recursiv de la radacina la dreapta, pana cand valoarea right este NULL\\
Asadar, timpul necesar este timpul parcurgerii inaltimii AVL -ului, mai exact O(log n)


\subsection{Succesor\slash predecesor cheie}

Aceasta presupun aflare celei mai mari chei y, mai mica decat cheia x , respectiv cele mai mici chei y, mai mare decat cheia x.
In cazul unui arbore binar de cautare (evident si in arborele AVL) succesorul cheii unui nod este nodul cu cea mai mica valoare din sub-arborele dreapt al cheii.
Daca sub-arborele drept nu exista, atunci elementul pe care il cautam se afla printre stramosii acestuia. Pentru a eficientiza cautarea succesorului, cand incepem parcurgerea de la radacina, chiar daca nu stim daca cheia are sau nu sub-arbore drept, vom salva si compara nodul cu cea mai mica valoare mai mare decat cheia data, asta pentru a nu parcurge inaltimea arborelui de doua ori. Daca nu exista un astfel de nod, cheia data nu are succesor.\\\\
Pentru predecesor operatiile se afla oarecum in oglinda, adica se va cauta nodul cu cheia cea mai mare din sub-arborele stang, iar daca nu exista se va lua nodul cu cea mai mica cheie din stramosii acestuia.
\\

In cel mai rau caz, parcurgem intreaga inaltime a arborelui, adica avem o complexitate de timp O(log n).


\subsection{Al k-lea element}

Pentru a afla al k-lea element, folosim traversarea in ordine (inorder traversal). Ideea este ca in timp ce traversam arborele, numaram nodurile, iar cand ajungem la elementul al k-lea se va returna cheia acestuia.

In cel mai rau caz, parcurgem intreaga lungime a arborelui, adica O(n), unde n este numarul de noduri.

\subsection{Cardinal }

Pentru afla cardinalul, am introdus o variabila care este incrementata in fiecare inserare si decrementata in fiecare stergere, iar functia aceasta pur si simplu returneaza valoarea variabilei.
Complexitatea functiei este asadar, O(1).

\subsection{Este In}

Cautarea este la fel ca intr-un arbore binar(asemanatoare cu cautarea binara intr-un array):\\
Pornim de la radacina si verificam daca cheia data este mai mare, mai mica sau egala decat aceasta.
Daca este egala, inseamna ca am gasit elementul, daca este mai mica, apelam recursiv pentru nodul stang, iar daca este mai mare apelam recursiv pentru nodul drept.
Daca in urma acestei parcurgeri nu gasim nodul, inseamna ca acesta nu exista.

In cel mai rau caz, parcurgem intreaga inaltime a arborelui AVL, adica O(log n).



\section{Motivatia Structurii}
Principalul motiv pentru care am ales arborele AVL este diferenta de complexitate fata de un arbore binar de cautare simplu (adusa de factorul de balansare) cand vine vorba de un input ordonat crescator/descrescator.\\
\noindent\makebox[\textwidth]{\includegraphics[width=\textwidth]{figura}}



\section{Avantaje si dezavantaje}
Arborii AVL au atat avantaje cat si dezavantaje.
Unul dintre principalele avantaje ar fi eficienta inalta atunci cand vine vorba de un numar mare de date de intrare care implica o multime de insertii, lucru care poate fi facut in complexitatea O(log n). \\Fata de un arbore de cautare normal, AVL-ul rezolva cazul marginal al datelor de intrare ordonate crescator/descrescator. \\Operatiile de cautare pot si ele sa fie facute cu aceeasi complexitate.
\\
Cu toate acestea, avem si cateva dezavantaje printre care
\begin{itemize}
\item Codul mult mai complex fata de codul unui arbore binar de cautare obisnuit, deoarece avem de tratat multe cazuri marginale.
\item Deoarece inaltimea trebuie sa fie mentinuta intr-un arbore AVL, se produc rotatii frecvente care,
vor necesita mai multe resurse pentru efectuarea lor in cazul unui numar mare de date de intrare.
\item Operatia de stergere de asemenea necesita multe resurse, deoarece implica multe schimbari de pointeri, in cel mai rau caz rotatiile fiind efectuate pana la radacina.

\end{itemize}


\section{Testare}


Am generat numerele cu ajutorul unui site (https://numbergenerator.org/) dupa ce ne-am umplut memoria cu un program in python.\\
Am ales sa testam urmatoarele multimi, pe 3 calculatoare diferite. Mai jos, vom enumera o medie a timpurilor de rulare. \\
\begin{enumerate}
\setlength\itemsep{-1.2em}
\item Noua numere random.\\
\item 100 de mii de numere in ordine aleatorie.\\
\item 100 de mii de numere in ordine aleatorie intre -100 de mii si +100 de mii\\
\item Numere de la -1.5 milioane la 1 milion ordonate crescator.\\
\item Numerele de la 1 la 5 milioane ordonate crescator.\\
\end{enumerate}



\begin{center}
Testele au fost realizate folosind functia "clock", din libraria bits/stdc++.h .\\
\noindent\makebox[\textwidth]{\includegraphics[width=\textwidth]{aaaa}}\\
\end{center}
In unele cazuri, functia analizata rula doar cateva zeci de instructiuni, si deoarece procesorul executa un numar foarte mare de instructiuni pe secunda, cronometrul returna 0 (zero) .\\


\begin{center}
Timpul de rulare\\
 \begin{tabular}{||c c c c c c||} 
 \hline
Operatia & Caz 1 & Caz 2 & Caz 3 & Caz 4 & Caz 5 \\ [0.5ex] 
 \hline
 Inserare & 	0 sec  & 	$\approx$0.073 sec & 	$\approx$0.095 sec & 	$\approx$2.2 sec & 	$\approx$5.4 sec\\ 
 \hline
 Stergere & 	0 sec  & 	$\approx$0.073 sec & 	$\approx$0.095 sec & 	$\approx$2.2 sec & 	$\approx$5.4 sec\\ 
 \hline
 Min\slash Max & 0 sec  & 	0 sec & 	0 sec & 	0 sec & 	0 sec\\ 
 \hline
 Suc\slash Pre & 	0 sec  & 	$\approx$0.0015 sec & $\approx$0.0015 sec & 	$\approx$0.0015 sec & 	$\approx$0.0015 sec\\ 
 \hline
 k element & 0 sec  & 	$\approx$0.002 sec & 	$\approx$0.0045 sec & 	$\approx$0.008 sec & 	$\approx$0.028 sec\\ 
 \hline
Cardinal & 0 sec  & 	0 sec & 	0 sec & 	0 sec & 	0 sec\\ 
 \hline
Este In &0 sec  & 	$\approx$0.0015 sec & 	$\approx$0.0015 sec & 	$\approx$0.0015 sec & 	$\approx$0.0015 sec\\ 
\hline

\end{tabular}
\end{center}
\subsection{Concluzie}
\noindent\makebox[\textwidth]{\includegraphics[width=\textwidth]{speed}}\\









\section{Sales Pitch}
Perfectly balanced, as all things should be!!\\
\noindent\makebox[\textwidth]{\includegraphics[width=\textwidth]{balanced}}\\



\end{document}